% Options for packages loaded elsewhere
\PassOptionsToPackage{unicode}{hyperref}
\PassOptionsToPackage{hyphens}{url}
%
\documentclass[
  english,
  man]{apa6}
\usepackage{lmodern}
\usepackage{amssymb,amsmath}
\usepackage{ifxetex,ifluatex}
\ifnum 0\ifxetex 1\fi\ifluatex 1\fi=0 % if pdftex
  \usepackage[T1]{fontenc}
  \usepackage[utf8]{inputenc}
  \usepackage{textcomp} % provide euro and other symbols
\else % if luatex or xetex
  \usepackage{unicode-math}
  \defaultfontfeatures{Scale=MatchLowercase}
  \defaultfontfeatures[\rmfamily]{Ligatures=TeX,Scale=1}
\fi
% Use upquote if available, for straight quotes in verbatim environments
\IfFileExists{upquote.sty}{\usepackage{upquote}}{}
\IfFileExists{microtype.sty}{% use microtype if available
  \usepackage[]{microtype}
  \UseMicrotypeSet[protrusion]{basicmath} % disable protrusion for tt fonts
}{}
\makeatletter
\@ifundefined{KOMAClassName}{% if non-KOMA class
  \IfFileExists{parskip.sty}{%
    \usepackage{parskip}
  }{% else
    \setlength{\parindent}{0pt}
    \setlength{\parskip}{6pt plus 2pt minus 1pt}}
}{% if KOMA class
  \KOMAoptions{parskip=half}}
\makeatother
\usepackage{xcolor}
\IfFileExists{xurl.sty}{\usepackage{xurl}}{} % add URL line breaks if available
\IfFileExists{bookmark.sty}{\usepackage{bookmark}}{\usepackage{hyperref}}
\hypersetup{
  pdftitle={Does Double-Hard Debias Keep its Promises? - A Replication Study},
  pdfauthor={Kristina Kobrock1, Meike Korsten1, \& Sonja Börgerding1},
  pdfkeywords={keywords},
  hidelinks,
  pdfcreator={LaTeX via pandoc}}
\urlstyle{same} % disable monospaced font for URLs
\usepackage{graphicx,grffile}
\makeatletter
\def\maxwidth{\ifdim\Gin@nat@width>\linewidth\linewidth\else\Gin@nat@width\fi}
\def\maxheight{\ifdim\Gin@nat@height>\textheight\textheight\else\Gin@nat@height\fi}
\makeatother
% Scale images if necessary, so that they will not overflow the page
% margins by default, and it is still possible to overwrite the defaults
% using explicit options in \includegraphics[width, height, ...]{}
\setkeys{Gin}{width=\maxwidth,height=\maxheight,keepaspectratio}
% Set default figure placement to htbp
\makeatletter
\def\fps@figure{htbp}
\makeatother
\setlength{\emergencystretch}{3em} % prevent overfull lines
\providecommand{\tightlist}{%
  \setlength{\itemsep}{0pt}\setlength{\parskip}{0pt}}
\setcounter{secnumdepth}{-\maxdimen} % remove section numbering
% Make \paragraph and \subparagraph free-standing
\ifx\paragraph\undefined\else
  \let\oldparagraph\paragraph
  \renewcommand{\paragraph}[1]{\oldparagraph{#1}\mbox{}}
\fi
\ifx\subparagraph\undefined\else
  \let\oldsubparagraph\subparagraph
  \renewcommand{\subparagraph}[1]{\oldsubparagraph{#1}\mbox{}}
\fi
% Manuscript styling
\usepackage{upgreek}
\captionsetup{font=singlespacing,justification=justified}

% Table formatting
\usepackage{longtable}
\usepackage{lscape}
% \usepackage[counterclockwise]{rotating}   % Landscape page setup for large tables
\usepackage{multirow}		% Table styling
\usepackage{tabularx}		% Control Column width
\usepackage[flushleft]{threeparttable}	% Allows for three part tables with a specified notes section
\usepackage{threeparttablex}            % Lets threeparttable work with longtable

% Create new environments so endfloat can handle them
% \newenvironment{ltable}
%   {\begin{landscape}\begin{center}\begin{threeparttable}}
%   {\end{threeparttable}\end{center}\end{landscape}}
\newenvironment{lltable}{\begin{landscape}\begin{center}\begin{ThreePartTable}}{\end{ThreePartTable}\end{center}\end{landscape}}

% Enables adjusting longtable caption width to table width
% Solution found at http://golatex.de/longtable-mit-caption-so-breit-wie-die-tabelle-t15767.html
\makeatletter
\newcommand\LastLTentrywidth{1em}
\newlength\longtablewidth
\setlength{\longtablewidth}{1in}
\newcommand{\getlongtablewidth}{\begingroup \ifcsname LT@\roman{LT@tables}\endcsname \global\longtablewidth=0pt \renewcommand{\LT@entry}[2]{\global\advance\longtablewidth by ##2\relax\gdef\LastLTentrywidth{##2}}\@nameuse{LT@\roman{LT@tables}} \fi \endgroup}

% \setlength{\parindent}{0.5in}
% \setlength{\parskip}{0pt plus 0pt minus 0pt}

% Overwrite redefinition of paragraph and subparagraph by the default LaTeX template
% See https://github.com/crsh/papaja/issues/292
\makeatletter
\renewcommand{\paragraph}{\@startsection{paragraph}{4}{\parindent}%
  {0\baselineskip \@plus 0.2ex \@minus 0.2ex}%
  {-1em}%
  {\normalfont\normalsize\bfseries\itshape\typesectitle}}

\renewcommand{\subparagraph}[1]{\@startsection{subparagraph}{5}{1em}%
  {0\baselineskip \@plus 0.2ex \@minus 0.2ex}%
  {-\z@\relax}%
  {\normalfont\normalsize\itshape\hspace{\parindent}{#1}\textit{\addperi}}{\relax}}
\makeatother

% \usepackage{etoolbox}
\makeatletter
\patchcmd{\HyOrg@maketitle}
  {\section{\normalfont\normalsize\abstractname}}
  {\section*{\normalfont\normalsize\abstractname}}
  {}{\typeout{Failed to patch abstract.}}
\patchcmd{\HyOrg@maketitle}
  {\section{\protect\normalfont{\@title}}}
  {\section*{\protect\normalfont{\@title}}}
  {}{\typeout{Failed to patch title.}}
\makeatother
\shorttitle{Replication: Wang et al. (2020)}
\keywords{keywords\newline\indent Word count: X}
\DeclareDelayedFloatFlavor{ThreePartTable}{table}
\DeclareDelayedFloatFlavor{lltable}{table}
\DeclareDelayedFloatFlavor*{longtable}{table}
\makeatletter
\renewcommand{\efloat@iwrite}[1]{\immediate\expandafter\protected@write\csname efloat@post#1\endcsname{}}
\makeatother
\usepackage{lineno}

\linenumbers
\usepackage{csquotes}
\ifxetex
  % Load polyglossia as late as possible: uses bidi with RTL langages (e.g. Hebrew, Arabic)
  \usepackage{polyglossia}
  \setmainlanguage[]{english}
\else
  \usepackage[shorthands=off,main=english]{babel}
\fi

\title{Does Double-Hard Debias Keep its Promises? - A Replication Study}
\author{Kristina Kobrock\textsuperscript{1}, Meike Korsten\textsuperscript{1}, \& Sonja Börgerding\textsuperscript{1}}
\date{}


\authornote{

The authors made the following contributions. Kristina Kobrock: müssen wir natürlich nicht ausfüllen; Meike Korsten: \ldots{}; Sonja Börgerding: \ldots{}.

}

\affiliation{\vspace{0.5cm}\textsuperscript{1} University of Osnabrück}

\begin{document}
\maketitle

Just edit the text! There are several websites that help with (r)markdown, e.g.~\url{https://rmarkdown.rstudio.com/}.
If you don't want to edit directly, but rather comment use:
Cite using: Wang et al.~(2020) said that\ldots{} or simply blablabla (Wang et al., 2020). References must be added to r-references.bib file.

\hypertarget{introduction}{%
\subsection{Introduction}\label{introduction}}

\begin{itemize}
\tightlist
\item
  shall we include an abstract?
  Recent research has shown that word embeddings derived from language corpora inherit human biases. The first seminal study on this topic was by Bolukbasi et al.~(2016) who used the word analogy task developed by Mikolov et al.~(2013) to prove that \enquote{Man is to Computer Programmer as Woman is to Homemaker} - a clearly gender biased analogical relation derived from a Google News embedding. Caliskan et al.~(2017) complement this finding with a large study on human-like semantic biases in Natural Language Processing (NLP) tools. They have shown in more general that human biases as exhibited in psychological studies using, for example, the Implicit Association Test (IAT) are learned by NLP algorithms designed to construct meaningful word representations. According to Zhao et al.~(2018a) these biases propagate to downstream tasks. As pre-trained word embeddings are often used for a lot of more complex NLP tasks and architectures of our everyday life, the biased embeddings bear the risk of proliferating and strengthening existing stereotypes in human cultures.
\end{itemize}

\hypertarget{debiasing-embeddings}{%
\subsubsection{Debiasing Embeddings}\label{debiasing-embeddings}}

But how can the problem of biased embeddings be solved? Several researchers have proposed post-processing techniques or algorithm modifications that promise to \enquote{debias} the word embeddings obtained by algorithms like word2vec or GloVe (e.g.~Bolukbasi et al., 2016; Kaneko \& Bollegala, 2019; Zhao et al., 2018b).
Bolukbasi et al.~(2016) have developed an algorithm called Hard Debias that is based on the idea of removing (biased) gender directions from the embedding whilst preserving desired gender relations. For example, the encoded gender of the words \enquote{king} and \enquote{queen} is given by definition. So the relation between \enquote{male} and \enquote{king} and between \enquote{female} and \enquote{queen} is desired. On the other hand, words like \enquote{nurse} and \enquote{doctor} also tend to exhibit relations to a specific gender even though the words do not have a gender-specific meaning but can be used for all genders. The proposed Hard Debias algorithm first identifies a gender subspace of the embedding that best captures the bias. Then a neutralizing step ensures that gender neutral words (like \enquote{nurse} and \enquote{doctor}) are indeed neutral, i.e.~zero, in the gender subspace. An equalizing step is then applied in order to ensure that useful relations apply to words from both genders and are not biased towards one gender anymore. For example, the word \enquote{babysit} should be equally distant from both \enquote{she} and \enquote{he}.
Wang et al.~(2020) build on Hard Debias in their newly proposed Double Hard Debias algorithm. The main idea is to not only remove the gender direction of the biased embedding, but also the frequency direction as it has been shown that word frequency has a significant impact on the geometry of the embedding space (e.g.~Gong et al., 2018; Mu, Bhat, \& Viswanath, 2018, more on this later).

\hypertarget{motivation}{%
\subsubsection{Motivation}\label{motivation}}

In this project, we aimed to replicate the debiasing method presented by Wang et al.~(2020) and to reproduce their experimental results as it is the most recent paper that proposes a post-processing technique for debiasing algorithms.
The reproduction of existing results is not only good practice in science, but it is also essential for gaining a deeper understanding on the methods used and it can help to validate existing results or to shed well-grounded doubt on them. Recent studies of reproducibility in the field of Computer Science (e.g.~Collberg, Proebsting, \& Warren, 2015) and NLP (e.g.~Fokkens et al., 2013; {\textbf{???}}; Belz, Agarwal, Shimorina, \& Reiter, 2021; Mieskes, 2017) explain why reproducibility endeavours are often failing and shed light on the exact nature of the \enquote{reproducibility crisis} (term coined by e.g.~Baker, 2015) in NLP research. Belz et al.~(2021), for example, report that the community's interest in topics of reproducibility have risen even though reproduction attempts still tend to fail due to problems like missing data, missing code and incomplete documentation (see also Fokkens et al., 2013; Mieskes, Fort, Névéol, Grouin, \& Cohen, 2019).
This project aims to make a contribution to the increasing body of reproducibility and replication attempts in NLP research.

\hypertarget{implementation}{%
\subsection{Implementation}\label{implementation}}

\begin{itemize}
\tightlist
\item
  task description
\item
  explanation of model and training choices
\end{itemize}

\hypertarget{datasets-and-preliminaries}{%
\subsubsection{Datasets and Preliminaries}\label{datasets-and-preliminaries}}

\hypertarget{hard-debias}{%
\subsubsection{Hard Debias}\label{hard-debias}}

\hypertarget{double-hard-debias}{%
\subsubsection{Double-Hard Debias}\label{double-hard-debias}}

\hypertarget{evaluation}{%
\subsection{Evaluation}\label{evaluation}}

\begin{itemize}
\tightlist
\item
  stick to the paper
\end{itemize}

\hypertarget{baselines}{%
\subsubsection{Baselines}\label{baselines}}

\hypertarget{evaluation-of-debiasing-performance}{%
\subsubsection{Evaluation of Debiasing Performance}\label{evaluation-of-debiasing-performance}}

\hypertarget{debiasing-in-downstream-applications}{%
\subsubsection{Debiasing in Downstream Applications}\label{debiasing-in-downstream-applications}}

\hypertarget{coreference-resolution}{%
\paragraph{Coreference Resolution}\label{coreference-resolution}}

\begin{itemize}
\tightlist
\item
  no replication possible, no code provided
\end{itemize}

\hypertarget{debiasing-at-embedding-level}{%
\subsubsection{Debiasing at Embedding Level}\label{debiasing-at-embedding-level}}

\hypertarget{the-word-embeddings-association-test-weat}{%
\paragraph{The Word Embeddings Association Test (WEAT)}\label{the-word-embeddings-association-test-weat}}

\hypertarget{neighborhood-metric}{%
\paragraph{Neighborhood Metric}\label{neighborhood-metric}}

\begin{itemize}
\tightlist
\item
  discuss that this is shady in the discussion part
\end{itemize}

\hypertarget{analysis-of-retaining-word-semantics}{%
\subsection{Analysis of Retaining Word Semantics}\label{analysis-of-retaining-word-semantics}}

\hypertarget{word-analogy}{%
\paragraph{Word Analogy}\label{word-analogy}}

\hypertarget{concept-categorization}{%
\paragraph{Concept Categorization}\label{concept-categorization}}

\hypertarget{discussion}{%
\subsection{Discussion}\label{discussion}}

\begin{itemize}
\tightlist
\item
  analysis of results and evaluation of performance evaluation
\item
  ablation studies (not applicable)
\item
  discuss the results and what could be (partly) replicated and what not
\end{itemize}

\hypertarget{conclusion}{%
\subsection{Conclusion}\label{conclusion}}

\newpage

\hypertarget{references}{%
\section{References}\label{references}}

\begingroup
\setlength{\parindent}{-0.5in}
\setlength{\leftskip}{0.5in}

\hypertarget{refs}{}
\leavevmode\hypertarget{ref-baker_2015}{}%
Baker, M. (2015). Reproducibility crisis: Blame it on the antibodies. \emph{Nature}, \emph{521}(7552), 274--276. \url{https://doi.org/https://doi.org/10.1038/521274a}

\leavevmode\hypertarget{ref-belz_2021}{}%
Belz, A., Agarwal, S., Shimorina, A., \& Reiter, E. (2021). \emph{A systematic review of reproducibility research in natural language processing}. Retrieved from \url{http://arxiv.org/abs/2103.07929}

\leavevmode\hypertarget{ref-bolukbasi_2016}{}%
Bolukbasi, T., Chang, K., Zou, J. Y., Saligrama, V., \& Kalai, A. (2016). Man is to computer programmer as woman is to homemaker? Debiasing word embeddings. \emph{CoRR}, \emph{abs/1607.06520}. Retrieved from \url{http://arxiv.org/abs/1607.06520}

\leavevmode\hypertarget{ref-caliskan_2017}{}%
Caliskan, A., Bryson, J. J., \& Narayanan, A. (2017). Semantics derived automatically from language corpora contain human-like biases. \emph{Science}, \emph{356}(6334), 183--186. \url{https://doi.org/10.1126/science.aal4230}

\leavevmode\hypertarget{ref-collberg_2015}{}%
Collberg, C., Proebsting, T., \& Warren, A. M. (2015). \emph{Repeatability and benefaction in computer systems research}. studie. Retrieved from \url{http://reproducibility.cs.arizona.edu/v2/RepeatabilityTR.pdf}

\leavevmode\hypertarget{ref-fokkens_2013}{}%
Fokkens, A., Erp, M. van, Postma, M., Pedersen, T., Vossen, P., \& Freire, N. (2013). Offspring from reproduction problems: What replication failure teaches us. \emph{Proceedings of the 51st annual meeting of the association for computational linguistics (volume 1: Long papers)}, 1691--1701. Sofia, Bulgaria: Association for Computational Linguistics. Retrieved from \url{https://www.aclweb.org/anthology/P13-1166}

\leavevmode\hypertarget{ref-gong_2018}{}%
Gong, C., He, D., Tan, X., Qin, T., Wang, L., \& Liu, T.-Y. (2018). FRAGE: Frequency-agnostic word representation. In S. Bengio, H. Wallach, H. Larochelle, K. Grauman, N. Cesa-Bianchi, \& R. Garnett (Eds.), \emph{Advances in neural information processing systems} (Vol. 31). Curran Associates, Inc. Retrieved from \url{https://proceedings.neurips.cc/paper/2018/file/e555ebe0ce426f7f9b2bef0706315e0c-Paper.pdf}

\leavevmode\hypertarget{ref-kaneko_2019}{}%
Kaneko, M., \& Bollegala, D. (2019). Gender-preserving debiasing for pre-trained word embeddings. \emph{Proceedings of the 57th annual meeting of the association for computational linguistics}, 1641--1650. Florence, Italy: Association for Computational Linguistics. \url{https://doi.org/10.18653/v1/P19-1160}

\leavevmode\hypertarget{ref-mieskes_2017}{}%
Mieskes, M. (2017). A quantitative study of data in the NLP community. \emph{Proceedings of the first ACL workshop on ethics in natural language processing}, 23--29. Valencia, Spain: Association for Computational Linguistics. \url{https://doi.org/10.18653/v1/W17-1603}

\leavevmode\hypertarget{ref-mieskes_2019}{}%
Mieskes, M., Fort, K., Névéol, A., Grouin, C., \& Cohen, K. B. (2019). NLP Community Perspectives on Replicability. \emph{Recent Advances in Natural Language Processing}. Varna, Bulgaria. Retrieved from \url{https://hal.archives-ouvertes.fr/hal-02282794}

\leavevmode\hypertarget{ref-mikolov2013MSR}{}%
Mikolov, T., Yih, W.-t., \& Zweig, G. (2013). Linguistic regularities in continuous space word representations. \emph{Proceedings of the 2013 conference of the north American chapter of the association for computational linguistics: Human language technologies}, 746--751. Atlanta, Georgia: Association for Computational Linguistics. Retrieved from \url{https://www.aclweb.org/anthology/N13-1090}

\leavevmode\hypertarget{ref-mu_2018}{}%
Mu, J., Bhat, S., \& Viswanath, P. (2018). \emph{All-but-the-top: Simple and effective postprocessing for word representations}. Retrieved from \url{http://arxiv.org/abs/1702.01417}

\leavevmode\hypertarget{ref-wang_2020}{}%
Wang, T., Lin, X. V., Rajani, N. F., McCann, B., Ordonez, V., \& Xiong, C. (2020). Double-hard debias: Tailoring word embeddings for gender bias mitigation. \emph{Association for computational linguistics (acl)}.

\leavevmode\hypertarget{ref-zhao_2018a}{}%
Zhao, J., Wang, T., Yatskar, M., Ordonez, V., \& Chang, K.-W. (2018a, January). \emph{Gender bias in coreference resolution: Evaluation and debiasing methods}. 15--20. \url{https://doi.org/10.18653/v1/N18-2003}

\leavevmode\hypertarget{ref-zhao_2018b}{}%
Zhao, J., Zhou, Y., Li, Z., Wang, W., \& Chang, K.-W. (2018b). Learning gender-neutral word embeddings. \emph{Proceedings of the 2018 conference on empirical methods in natural language processing}, 4847--4853. Brussels, Belgium: Association for Computational Linguistics. \url{https://doi.org/10.18653/v1/D18-1521}

\endgroup


\end{document}
